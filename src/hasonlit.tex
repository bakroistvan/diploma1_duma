%----------------------------------------------------------------------------
\chapter{Összehasonlítás}
%----------------------------------------------------------------------------
	A programot a különböző eszközökön futtatva a \ref{table:results}. táblázatban látható futási
	eredményeket produkálta. A táblázatban látható futási idők 100 futási idő átlaga.
	A táblázathoz felvettem egy fiktív mérőszámot (teljesítmény tényező) a különböző architektúra
	összehasonlítására.
	A mérőszám értéke minél kissebb, annál gyorsabban hajtódik végre egy utasítás.
	\begin{table}[H]
	\footnotesize
	\centering
	\caption[Eszközök futási idejének összehasonlítása]{Az eszközök erőforrásainak és a rajta futtatott
	programok futási idejének összehasonlítása.}
	\label{table:results}
	\setlength{\extrarowheight}{8pt}
% 		\begin{tabular}{ l | r | r | r | r}
% 		 & Intel Core i5 & nVidia GT330M & Intel Xeon & Xeon PHI \\ \hline
% 		MAX COMPUTE UNITS & $4$ & $6$ & $8$ & $224$\\
% 		MAX CLOCK FREQUENCY & 2400 & 1265 & 3000 & 1100\\
% 		MAX WORK GROUP\_SIZE & $8192$ & $512$ & $8192$ & $8192$ \\ \hline\hline
% 		GLOBAL MEM SIZE & $\sim 4Gbyte$ & $1Gbyte$ & $8Gbyte$ & $\sim 4.5Gbyte$\\
% 		LOCAL MEM SIZE & $32 Kbyte$ & $16 Kbyte$ & $32 Kbyte$ & $32 Kbyte$\\ \hline\hline
% 		Futási idő &
% 		\end{tabular}
	\begin{tabular}{ l | r | r}
		 & Intel Core i5 M520 & nVidia GT330M \\ \hline
		MAX COMPUTE UNITS $[1]$ & $4$ & $6$\\
		MAX CLOCK FREQUENCY $[\mathrm{MHz}]$ & 2400 & 1265 \\
		MAX WORK GROUP\_SIZE & $8192$ & $512$ \\ \hline\hline
		GLOBAL MEM SIZE & $\sim 4\ \mathrm{GByte}$ & $\sim 1\ \mathrm{GByte}$\\
		%MAX\_CONSTANT\_BUFFER\_SIZE & $131072$ & $65536$ & $131072$ & $131072$\\
		LOCAL MEM SIZE & $32\ \mathrm{KByte}$ & $16\ \mathrm{KByte}$\\ \hline\hline
% 		Kép mérete & $1024\times 1024$ & $1024\times 1024$ \\ 
		Futási idő $(T)$ & $478.71\ ms$ & $191.94\ ms$\\
		Teljesítmény tényező $\left(P=\frac{1}{\mathrm{UNITS}\times \mathrm{FREQUENCY}}\right)$ &
		$104.16\cdot 10^{-6}$ & $131.75\cdot 10^{-6}$\\
		Fajlagos utasításszám $(T/P)$ &  $4.59\cdot 10^3$ & $1.45\cdot 10^3$ 
	\end{tabular}
	\end{table}
	\noindent Látható, hogy a GPU-n való futtatás közel $3\times$ gyorsulást jelent.
	Ezt két dolognak tudom be:
	\begin{itemize}
		\item \emph{Memória:} A CPU memóriája DDR3 @ 1066 MHz 64 bites busz szélességgel, míg a GPU
		memóriája GDDR3 @ 1066 MHz 128 bites busz szélességgel,
		\item \emph{Processzor mag:} A CPU 4 compute unit-al rendelkezik, ami 4 szálra, ami 2
		processzormagra az Intel HyperThread technológiájával képeződik le, míg a GPU 6 compute unit-al
		rendelkezik, ami 48 CUDA core-ra képeződik le.
	\end{itemize}
	Továbbá figyelembe kell venni, hogy a program futása a többi programmal konkurrensen történik, CPU
	esetén az operációs rendszerrel, GPU esetén a megjelenítéssel.
	


