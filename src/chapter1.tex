%----------------------------------------------------------------------------
\chapter{A Poisson egyenlet bemutatása}
%----------------------------------------------------------------------------
\section{A Poisson egyenlet}
%----------------------------------------------------------------------------

A Poisson egyenlet egy elliptikus parciális differenciál egyenlet.
Az elektrosztatika, a mechanika, hővezetés és a fizika sok problémája erre az egyenletre vezet.\\
Általános alakja, ahol $\varphi$ és $f$ valós vagy komplex értékű vektorfüggvények:
\begin{align}
\nabla^2 \varphi &= f\\
\left( \frac{\partial^2}{\partial x^2} + \frac{\partial^2}{\partial y^2} + \frac{\partial^2}{\partial z^2} \right)\varphi(x,y,z) &= f(x,y,z).
\end{align}
$f = 0$ esetén a Laplace egyenletre redukálódik.
Általánosan elmondható, hogy a megoldást $\varphi$ nem kívánjuk tudni a teljes téren, csupán
egy jól meghatározótt tartományon belül ($\Omega \in \mathbb{R}$).
A differenciálegyenlet egyértelmű megoldásához a következő peremfeltételek érvényre juttatására van szükség:
\begin{itemize}
\item Dirichlet: $\varphi(x,y,z) = f_0(x,y,z) \quad (x,y,z) \in \partial\Omega$
\item Neumann $\frac{\partial\varphi}{\partial n}(x,y,z) = g_0(x,y,z) \quad (x,y,z) \in \partial\Omega$
\item stb.
\end{itemize}
Az így összeállt probléma már megoldható.
Megoldani szeparálással és más analitikus módszerrel illetve a következő numerikus módszerekkel lehetséges:
\begin{itemize}
\item Véges elemek
\item Véges differenciák ( ha a megoldás feltehetően elég sima )
\end{itemize}
\newpage

%----------------------------------------------------------------------------
\section{Az elektrosztatikus probléma}
%----------------------------------------------------------------------------
A dolgozat során az elektrosztatikus téret leíró, Poisson (Laplace) egyenletre
redukálódó egyenlet megoldását keresem.
Ismeretes, hogy ilyenkor áram nem folyik és a gerjesztő menniységek állandóak, 
azaz $\vec{J} = 0$ és $\frac{\partial}{\partial t} \equiv 0$. Ezen feltételekkel a Maxwell-egyenletek a következő alakot veszik fel:
\begin{align}
rot \vec{E} &= 0\\
div \vec{D} &= \rho\\
D &= \varepsilon \vec{E}
\end{align}
Az örvénymentes villamos tér felírhatő skalárpotenciál gradienseként:
\begin{equation}
\vec{E} = - grad \varphi
\end{equation}
Ezzel a lépéssel jutunk az elektrosztatika Poisson egyenletére:
\begin{align}
-div(\varepsilon grad \varphi ) &= \rho
\end{align}
Ha homogén az anyag akkor:
\begin{align}
	\nabla^2\varphi  &= \frac{\rho}{\varepsilon}
\end{align}






























